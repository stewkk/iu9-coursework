% !TeX TXS-program:bibliography = txs:///biber
\documentclass[14pt, russian]{scrartcl}
\let\counterwithout\relax
\let\counterwithin\relax
%\usepackage{lmodern}
\usepackage{float}
\usepackage{xcolor}
\usepackage{extsizes}
\usepackage{subfig}
\usepackage[export]{adjustbox}
\usepackage{tocvsec2} % возможность менять учитываемую глубину разделов в оглавлении
\usepackage[subfigure]{tocloft}
\usepackage[newfloat]{minted}
\captionsetup[listing]{position=top}

\AtBeginEnvironment{figure}{\vspace{0.5cm}}
\AtBeginEnvironment{table}{\vspace{0.5cm}}
\AtBeginEnvironment{listing}{\vspace{0.5cm}}
\AtBeginEnvironment{algorithm}{\vspace{0.5cm}}
\AtBeginEnvironment{minted}{\vspace{-0.5cm}}

\usepackage{fancyvrb}
\usepackage{ulem,bm,mathrsfs,ifsym} %зачеркивания, особо жирный стиль и RSFS начертание
\usepackage{sectsty} % переопределение стилей подразделов
%%%%%%%%%%%%%%%%%%%%%%%

%%% Поля и разметка страницы %%%
\usepackage{pdflscape}                              % Для включения альбомных страниц
\usepackage{geometry}                               % Для последующего задания полей
\geometry{a4paper,tmargin=2cm,bmargin=2cm,lmargin=3cm,rmargin=1cm} % тоже самое, но лучше

%%% Математические пакеты %%%
\usepackage{amsthm,amsfonts,amsmath,amssymb,amscd}  % Математические дополнения от AMS
\usepackage{mathtools}                              % Добавляет окружение multlined
\usepackage[perpage]{footmisc}
%\usepackage{times}

%%%% Установки для размера шрифта 14 pt %%%%
%% Формирование переменных и констант для сравнения (один раз для всех подключаемых файлов)%%
%% должно располагаться до вызова пакета fontspec или polyglossia, потому что они сбивают его работу
%\newlength{\curtextsize}
%\newlength{\bigtextsize}
%\setlength{\bigtextsize}{13pt}
\KOMAoptions{fontsize=14pt}

\makeatletter
\def\showfontsize{\f@size{} point}
\makeatother

%\makeatletter
%\show\f@size                                       % неплохо для отслеживания, но вызывает стопорение процесса, если документ компилируется без команды  -interaction=nonstopmode 
%\setlength{\curtextsize}{\f@size pt}
%\makeatother

%шрифт times
\usepackage{tempora}
%\usepackage{pscyr}
%\setmainfont[Ligatures={TeX,Historic}]{Times New Roman}

   %%% Решение проблемы копирования текста в буфер кракозябрами
%    \input glyphtounicode.tex
%    \input glyphtounicode-cmr.tex %from pdfx package
%    \pdfgentounicode=1
    \usepackage{cmap}                               % Улучшенный поиск русских слов в полученном pdf-файле
    \usepackage[T1]{fontenc}                       % Поддержка русских букв
    \usepackage[utf8]{inputenc}                     % Кодировка utf8
    \usepackage[english, main=russian]{babel}            % Языки: русский, английский
%   \IfFileExists{pscyr.sty}{\usepackage{pscyr}}{}  % Красивые русские шрифты
%\renewcommand{\rmdefault}{ftm}
%%% Оформление абзацев %%%
\usepackage{indentfirst}                            % Красная строка
%\usepackage{eskdpz}

%%% Таблицы %%%
\usepackage{longtable}                              % Длинные таблицы
\usepackage{multirow,makecell,array}                % Улучшенное форматирование таблиц
\usepackage{booktabs}                               % Возможность оформления таблиц в классическом книжном стиле (при правильном использовании не противоречит ГОСТ)

%%% Общее форматирование
\usepackage{soulutf8}                               % Поддержка переносоустойчивых подчёркиваний и зачёркиваний
\usepackage{icomma}                                 % Запятая в десятичных дробях



%%% Изображения %%%
\usepackage{graphicx}                               % Подключаем пакет работы с графикой
\usepackage{wrapfig}

\usepackage{tikz}
\usetikzlibrary{shapes.misc}
\usetikzlibrary{trees}

%%% Списки %%%
\usepackage{enumitem}

%%% Подписи %%%
\usepackage{caption}                                % Для управления подписями (рисунков и таблиц) % Может управлять номерами рисунков и таблиц с caption %Иногда может управлять заголовками в списках рисунков и таблиц
%% Использование:
%\begin{table}[h!]\ContinuedFloat - чтобы не переключать счетчик
%\captionsetup{labelformat=continued}% должен стоять до самого caption
%\caption{}
% либо ручками \caption*{Продолжение таблицы~\ref{...}.} :)

%%% Интервалы %%%
\addto\captionsrussian{%
  \renewcommand{\listingname}{Листинг}%
}
%%% Счётчики %%%
\usepackage[figure,table,section]{totalcount}               % Счётчик рисунков и таблиц
\DeclareTotalCounter{lstlisting}
\usepackage{totcount}                               % Пакет создания счётчиков на основе последнего номера подсчитываемого элемента (может требовать дважды компилировать документ)
\usepackage{totpages}                               % Счётчик страниц, совместимый с hyperref (ссылается на номер последней страницы). Желательно ставить последним пакетом в преамбуле

%%% Продвинутое управление групповыми ссылками (пока только формулами) %%%
%% Кодировки и шрифты %%%

%   \newfontfamily{\cyrillicfont}{Times New Roman}
%   \newfontfamily{\cyrillicfonttt}{CMU Typewriter Text}
	%\setmainfont{Times New Roman}
	%\newfontfamily\cyrillicfont{Times New Roman}
	%\setsansfont{Times New Roman}                    %% задаёт шрифт без засечек
%	\setmonofont{Liberation Mono}               %% задаёт моноширинный шрифт
%    \IfFileExists{pscyr.sty}{\renewcommand{\rmdefault}{ftm}}{}
%%% Интервалы %%%
%linespread-реализация ближе к реализации полуторного интервала в ворде.
%setspace реализация заточена под шрифты 10, 11, 12pt, под остальные кегли хуже, но всё же ближе к типографской классике. 
\linespread{1.3}                    % Полуторный интервал (ГОСТ Р 7.0.11-2011, 5.3.6)
%\renewcommand{\@biblabel}[1]{#1}

%%% Гиперссылки %%%
\usepackage{hyperref}

%%% Выравнивание и переносы %%%
\sloppy                             % Избавляемся от переполнений
\clubpenalty=10000                  % Запрещаем разрыв страницы после первой строки абзаца
\widowpenalty=10000                 % Запрещаем разрыв страницы после последней строки абзаца

\makeatletter % малые заглавные, small caps shape
\let\@@scshape=\scshape
\renewcommand{\scshape}{%
  \ifnum\strcmp{\f@series}{bx}=\z@
    \usefont{T1}{cmr}{bx}{sc}%
  \else
    \ifnum\strcmp{\f@shape}{it}=\z@
      \fontshape{scsl}\selectfont
    \else
      \@@scshape
    \fi
  \fi}
\makeatother

%%% Подписи %%%
%\captionsetup{%
%singlelinecheck=off,                % Многострочные подписи, например у таблиц
%skip=2pt,                           % Вертикальная отбивка между подписью и содержимым рисунка или таблицы определяется ключом
%justification=centering,            % Центрирование подписей, заданных командой \caption
%}
%%%        Подключение пакетов                 %%%
\usepackage{ifthen}                 % добавляет ifthenelse
%%% Инициализирование переменных, не трогать!  %%%
\newcounter{intvl}
\newcounter{otstup}
\newcounter{contnumeq}
\newcounter{contnumfig}
\newcounter{contnumtab}
\newcounter{pgnum}
\newcounter{bibliosel}
\newcounter{chapstyle}
\newcounter{headingdelim}
\newcounter{headingalign}
\newcounter{headingsize}
\newcounter{tabcap}
\newcounter{tablaba}
\newcounter{tabtita}
%%%%%%%%%%%%%%%%%%%%%%%%%%%%%%%%%%%%%%%%%%%%%%%%%%

%%% Область упрощённого управления оформлением %%%

%% Интервал между заголовками и между заголовком и текстом
% Заголовки отделяют от текста сверху и снизу тремя интервалами (ГОСТ Р 7.0.11-2011, 5.3.5)
\setcounter{intvl}{3}               % Коэффициент кратности к размеру шрифта

%% Отступы у заголовков в тексте
\setcounter{otstup}{0}              % 0 --- без отступа; 1 --- абзацный отступ

%% Нумерация формул, таблиц и рисунков
\setcounter{contnumeq}{1}           % Нумерация формул: 0 --- пораздельно (во введении подряд, без номера раздела); 1 --- сквозная нумерация по всей диссертации
\setcounter{contnumfig}{1}          % Нумерация рисунков: 0 --- пораздельно (во введении подряд, без номера раздела); 1 --- сквозная нумерация по всей диссертации
\setcounter{contnumtab}{1}          % Нумерация таблиц: 0 --- пораздельно (во введении подряд, без номера раздела); 1 --- сквозная нумерация по всей диссертации

%% Оглавление
\setcounter{pgnum}{0}               % 0 --- номера страниц никак не обозначены; 1 --- Стр. над номерами страниц (дважды компилировать после изменения)

%% Библиография
\setcounter{bibliosel}{1}           % 0 --- встроенная реализация с загрузкой файла через движок bibtex8; 1 --- реализация пакетом biblatex через движок biber

%% Текст и форматирование заголовков
\setcounter{chapstyle}{1}           % 0 --- разделы только под номером; 1 --- разделы с названием "Глава" перед номером
\setcounter{headingdelim}{1}        % 0 --- номер отделен пропуском в 1em или \quad; 1 --- номера разделов и приложений отделены точкой с пробелом, подразделы пропуском без точки; 2 --- номера разделов, подразделов и приложений отделены точкой с пробелом.

%% Выравнивание заголовков в тексте
\setcounter{headingalign}{0}        % 0 --- по центру; 1 --- по левому краю

%% Размеры заголовков в тексте
\setcounter{headingsize}{0}         % 0 --- по ГОСТ, все всегда 14 пт; 1 --- пропорционально изменяющийся размер в зависимости от базового шрифта

%% Подпись таблиц
\setcounter{tabcap}{0}              % 0 --- по ГОСТ, номер таблицы и название разделены тире, выровнены по левому краю, при необходимости на нескольких строках; 1 --- подпись таблицы не по ГОСТ, на двух и более строках, дальнейшие настройки: 
%Выравнивание первой строки, с подписью и номером
\setcounter{tablaba}{2}             % 0 --- по левому краю; 1 --- по центру; 2 --- по правому краю
%Выравнивание строк с самим названием таблицы
\setcounter{tabtita}{1}             % 0 --- по левому краю; 1 --- по центру; 2 --- по правому краю

%%% Рисунки %%%
\DeclareCaptionLabelSeparator*{emdash}{~--- }             % (ГОСТ 2.105, 4.3.1)
\captionsetup[figure]{labelsep=emdash,font=onehalfspacing,position=bottom}

%%% Таблицы %%%
\ifthenelse{\equal{\thetabcap}{0}}{%
    \newcommand{\tabcapalign}{\raggedright}  % по левому краю страницы или аналога parbox
}

\ifthenelse{\equal{\thetablaba}{0} \AND \equal{\thetabcap}{1}}{%
    \newcommand{\tabcapalign}{\raggedright}  % по левому краю страницы или аналога parbox
}

\ifthenelse{\equal{\thetablaba}{1} \AND \equal{\thetabcap}{1}}{%
    \newcommand{\tabcapalign}{\centering}    % по центру страницы или аналога parbox
}

\ifthenelse{\equal{\thetablaba}{2} \AND \equal{\thetabcap}{1}}{%
    \newcommand{\tabcapalign}{\raggedleft}   % по правому краю страницы или аналога parbox
}

\ifthenelse{\equal{\thetabtita}{0} \AND \equal{\thetabcap}{1}}{%
    \newcommand{\tabtitalign}{\raggedright}  % по левому краю страницы или аналога parbox
}

\ifthenelse{\equal{\thetabtita}{1} \AND \equal{\thetabcap}{1}}{%
    \newcommand{\tabtitalign}{\centering}    % по центру страницы или аналога parbox
}

\ifthenelse{\equal{\thetabtita}{2} \AND \equal{\thetabcap}{1}}{%
    \newcommand{\tabtitalign}{\raggedleft}   % по правому краю страницы или аналога parbox
}

\DeclareCaptionFormat{tablenocaption}{\tabcapalign #1\strut}        % Наименование таблицы отсутствует
\ifthenelse{\equal{\thetabcap}{0}}{%
    \DeclareCaptionFormat{tablecaption}{\tabcapalign #1#2#3}
    \captionsetup[table]{labelsep=emdash}                       % тире как разделитель идентификатора с номером от наименования
}{%
    \DeclareCaptionFormat{tablecaption}{\tabcapalign #1#2\par%  % Идентификатор таблицы на отдельной строке
        \tabtitalign{#3}}                                       % Наименование таблицы строкой ниже
    \captionsetup[table]{labelsep=space}                        % пробельный разделитель идентификатора с номером от наименования
}
\captionsetup[table]{format=tablecaption,singlelinecheck=off,font=onehalfspacing,position=top,skip=-5pt}  % многострочные наименования и прочее
\DeclareCaptionLabelFormat{continued}{Продолжение таблицы~#2}
\setlength{\belowcaptionskip}{.2cm}
\setlength{\intextsep}{0ex}

%%% Подписи подрисунков %%%
\renewcommand{\thesubfigure}{\asbuk{subfigure}}           % Буквенные номера подрисунков
\captionsetup[subfigure]{font={normalsize},               % Шрифт подписи названий подрисунков (не отличается от основного)
    labelformat=brace,                                    % Формат обозначения подрисунка
    justification=centering,                              % Выключка подписей (форматирование), один из вариантов            
}
%\DeclareCaptionFont{font12pt}{\fontsize{12pt}{13pt}\selectfont} % объявляем шрифт 12pt для использования в подписях, тут же надо интерлиньяж объявлять, если не наследуется
%\captionsetup[subfigure]{font={font12pt}}                 % Шрифт подписи названий подрисунков (всегда 12pt)

%%% Настройки гиперссылок %%%

\definecolor{linkcolor}{rgb}{0.0,0,0}
\definecolor{citecolor}{rgb}{0,0.0,0}
\definecolor{urlcolor}{rgb}{0,0,0}

\hypersetup{
    linktocpage=true,           % ссылки с номера страницы в оглавлении, списке таблиц и списке рисунков
%    linktoc=all,                % both the section and page part are links
%    pdfpagelabels=false,        % set PDF page labels (true|false)
    plainpages=true,           % Forces page anchors to be named by the Arabic form  of the page number, rather than the formatted form
    colorlinks,                 % ссылки отображаются раскрашенным текстом, а не раскрашенным прямоугольником, вокруг текста
    linkcolor={linkcolor},      % цвет ссылок типа ref, eqref и подобных
    citecolor={citecolor},      % цвет ссылок-цитат
    urlcolor={urlcolor},        % цвет гиперссылок
    pdflang={ru},
}
\urlstyle{same}
%%% Шаблон %%%
%\DeclareRobustCommand{\todo}{\textcolor{red}}       % решаем проблему превращения названия цвета в результате \MakeUppercase, http://tex.stackexchange.com/a/187930/79756 , \DeclareRobustCommand protects \todo from expanding inside \MakeUppercase
\setlength{\parindent}{2.5em}                       % Абзацный отступ. Должен быть одинаковым по всему тексту и равен пяти знакам (ГОСТ Р 7.0.11-2011, 5.3.7).

%%% Списки %%%
% Используем дефис для ненумерованных списков (ГОСТ 2.105-95, 4.1.7)
%\renewcommand{\labelitemi}{\normalfont\bfseries~{---}} 
\renewcommand{\labelitemi}{\bfseries~{---}} 
\setlist{nosep,%                                    % Единый стиль для всех списков (пакет enumitem), без дополнительных интервалов.
    labelindent=\parindent,leftmargin=*%            % Каждый пункт, подпункт и перечисление записывают с абзацного отступа (ГОСТ 2.105-95, 4.1.8)
}
%%%%%%%%%%%%%%%%%%%%%%
%\usepackage{xltxtra} % load xunicode

\usepackage{ragged2e}
\usepackage[explicit]{titlesec}
\usepackage{placeins}
\usepackage{xparse}
\usepackage{csquotes}

\usepackage{listingsutf8}
\usepackage{url} %пакеты расширений
\usepackage{algorithm, algorithmicx}
\usepackage[noend]{algpseudocode}
\usepackage{blkarray}
\usepackage{chngcntr}
\usepackage{tabularx}
\usepackage[backend=biber, 
    bibstyle=gost-numeric,
    citestyle=nature]{biblatex}
\newcommand*\template[1]{\text{<}#1\text{>}}
\addbibresource{biblio.bib}
  
\titleformat{name=\section,numberless}[block]{\normalfont\Large\centering}{}{0em}{#1}
\titleformat{\section}[block]{\normalfont\Large\bfseries\raggedright}{}{0em}{\thesection\hspace{0.25em}#1}
\titleformat{\subsection}[block]{\normalfont\Large\bfseries\raggedright}{}{0em}{\thesubsection\hspace{0.25em}#1}
\titleformat{\subsubsection}[block]{\normalfont\large\bfseries\raggedright}{}{0em}{\thesubsubsection\hspace{0.25em}#1}

\let\Algorithm\algorithm
\renewcommand\algorithm[1][]{\Algorithm[#1]\setstretch{1.5}}
%\renewcommand{\listingscaption}{Листинг}

\usepackage{pifont}
\usepackage{calc}
\usepackage{suffix}
\usepackage{csquotes}
\DeclareQuoteStyle{russian}
    {\guillemotleft}{\guillemotright}[0.025em]
    {\quotedblbase}{\textquotedblleft}
\ExecuteQuoteOptions{style=russian}
\newcommand{\enq}[1]{\enquote{#1}}  
\newcommand{\eng}[1]{\begin{english}#1\end{english}}
% Подчиненные счетчики в окружениях http://old.kpfu.ru/journals/izv_vuz/arch/sample1251.tex
\newcounter{cTheorem} 
\newcounter{cDefinition}
\newcounter{cConsequent}
\newcounter{cExample}
\newcounter{cLemma}
\newcounter{cConjecture}
\newtheorem{Theorem}{Теорема}[cTheorem]
\newtheorem{Definition}{Определение}[cDefinition]
\newtheorem{Consequent}{Следствие}[cConsequent]
\newtheorem{Example}{Пример}[cExample]
\newtheorem{Lemma}{Лемма}[cLemma]
\newtheorem{Conjecture}{Гипотеза}[cConjecture]

\renewcommand{\theTheorem}{\arabic{Theorem}}
\renewcommand{\theDefinition}{\arabic{Definition}}
\renewcommand{\theConsequent}{\arabic{Consequent}}
\renewcommand{\theExample}{\arabic{Example}}
\renewcommand{\theLemma}{\arabic{Lemma}}
\renewcommand{\theConjecture}{\arabic{Conjecture}}
%\makeatletter
\NewDocumentCommand{\Newline}{}{\text{\\}}
\newcommand{\sequence}[2]{\ensuremath \left(#1,\ \dots,\ #2\right)}

\definecolor{mygreen}{rgb}{0,0.6,0}
\definecolor{mygray}{rgb}{0.5,0.5,0.5}
\definecolor{mymauve}{rgb}{0.58,0,0.82}
\renewcommand{\listalgorithmname}{Список алгоритмов}
\floatname{algorithm}{Листинг}
\renewcommand{\lstlistingname}{Листинг}
\renewcommand{\thealgorithm}{\arabic{algorithm}}

\newcommand{\refAlgo}[1]{(листинг \ref{#1})}
\newcommand{\refImage}[1]{(рисунок \ref{#1})}

\renewcommand{\theenumi}{\arabic{enumi}.}% Меняем везде перечисления на цифра.цифра	
\renewcommand{\labelenumi}{\arabic{enumi}.}% Меняем везде перечисления на цифра.цифра
\renewcommand{\theenumii}{\arabic{enumii}}% Меняем везде перечисления на цифра.цифра
\renewcommand{\labelenumii}{(\arabic{enumii})}% Меняем везде перечисления на цифра.цифра
\renewcommand{\theenumiii}{\roman{enumiii}}% Меняем везде перечисления на цифра.цифра
\renewcommand{\labelenumiii}{(\roman{enumiii})}% Меняем везде перечисления на цифра.цифра
%\newfontfamily\AnkaCoder[Path=src/fonts/]{AnkaCoder-r.ttf}
\renewcommand{\labelitemi}{---}
\renewcommand{\labelitemii}{---}

%\usepackage{courier}

\lstdefinelanguage{Refal}{
  alsodigit = {.,<,>},
  morekeywords = [1]{$ENTRY},
  morekeywords = [2]{Go, Put, Get, Open, Close, Arg, Add, Sub, Mul, Div, Symb, Explode, Implode},
  %keyword4
  morekeywords = [3]{<,>},
  %keyword5
  morekeywords = [4]{e.,t.,s.},
  sensitive = true,
  morecomment = [l]{*},
  morecomment = [s]{/*}{*/},
  commentstyle = \color{mygreen},
  morestring = [b]",
  morestring = [b]',
  stringstyle = \color{purple}
}

\makeatletter
\def\p@subsection{}
\def\p@subsubsection{\thesection\,\thesubsection\,}
\makeatother
\newcommand{\prog}[1]{{\ttfamily\small#1}}
\lstset{ %
  backgroundcolor=\color{white},   % choose the background color; you must add \usepackage{color} or \usepackage{xcolor}
  basicstyle=\ttfamily\footnotesize, 
  %basicstyle=\footnotesize\AnkaCoder,        % the size of the fonts that are used for the code
  breakatwhitespace=false,         % sets if automatic breaks shoulbd only happen at whitespace
  breaklines=true,                 % sets automatic line breaking
  captionpos=top,                    % sets the caption-position to bottom
  commentstyle=\color{mygreen},    % comment style
  deletekeywords={...},            % if you want to delete keywords from the given language
  escapeinside={\%*}{*)},          % if you want to add LaTeX within your code
  extendedchars=true,              % lets you use non-ASCII characters; for 8-bits encodings only, does not work with UTF-8
  inputencoding=utf8,
  frame=single,                    % adds a frame around the code
  keepspaces=true,                 % keeps spaces in text, useful for keeping indentation of code (possibly needs columns=flexible)
  keywordstyle=\bf,       % keyword style
  language=Refal,                    % the language of the code
  morekeywords={<,>,$ENTRY,Go,Arg, Open, Close, e., s., t., Get, Put}, 
  							       % if you want to add more keywords to the set
  numbers=left,                    % where to put the line-numbers; possible values are (none, left, right)
  numbersep=5pt,                   % how far the line-numbers are from the code
  xleftmargin=25pt,
  xrightmargin=25pt,
  numberstyle=\small\color{black}, % the style that is used for the line-numbers
  rulecolor=\color{black},         % if not set, the frame-color may be changed on line-breaks within not-black text (e.g. comments (green here))
  showspaces=false,                % show spaces everywhere adding particular underscores; it overrides 'showstringspaces'
  showstringspaces=false,          % underline spaces within strings only
  showtabs=false,                  % show tabs within strings adding particular underscores
  stepnumber=1,                    % the step between two line-numbers. If it's 1, each line will be numbered
  stringstyle=\color{mymauve},     % string literal style
  tabsize=8,                       % sets default tabsize to 8 spaces
  title=\lstname                   % show the filename of files included with \lstinputlisting; also try caption instead of title
}
\newcommand{\anonsection}[1]{\cleardoublepage
\phantomsection
\addcontentsline{toc}{section}{\protect\numberline{}#1}
\section*{#1}\vspace*{2.5ex} % По госту положены 3 пустые строки после заголовка ненумеруемого раздела
}
\newcommand{\sectionbreak}{\clearpage}
\renewcommand{\sectionfont}{\normalsize} % Сбиваем стиль оглавления в стандартный
\renewcommand{\cftsecleader}{\cftdotfill{\cftdotsep}} % Точки в оглавлении напротив разделов

\renewcommand{\cftsecfont}{\normalfont\large} % Переключение на times в содержании
\renewcommand{\cftsubsecfont}{\normalfont\large} % Переключение на times в содержании

\usepackage{caption} 
%\captionsetup[table]{justification=raggedleft} 
%\captionsetup[figure]{justification=centering,labelsep=endash}
\usepackage{amsmath}    % \bar    (матрицы и проч. ...)
\usepackage{amsfonts}   % \mathbb (символ для множества действительных чисел и проч. ...)
\usepackage{mathtools}  % \abs, \norm
    \DeclarePairedDelimiter\abs{\lvert}{\rvert} % операция модуля
    \DeclarePairedDelimiter\norm{\lVert}{\rVert} % операция нормы
\DeclareTextCommandDefault{\textvisiblespace}{%
  \mbox{\kern.06em\vrule \@height.3ex}%
  \vbox{\hrule \@width.3em}%
  \hbox{\vrule \@height.3ex}}    
\newsavebox{\spacebox}
\begin{lrbox}{\spacebox}
\verb*! !
\end{lrbox}
\newcommand{\aspace}{\usebox{\spacebox}}
\DeclareTotalCounter{listing}

\makeatletter
\renewcommand*{\p@subsubsection}{}
\makeatother

\makeatletter
\AddToHook{begindocument/before}{\@ifpackageloaded{minted}{\removefromtoclist[float]{lol}}{}}
\makeatother

\begin{document}
\sloppy

\def\figurename{Рисунок}

\begin{titlepage}
	\thispagestyle{empty}
	\newpage

	\vspace*{-30pt}
	\hspace{-45pt}
	\begin{minipage}{0.17\textwidth}
		\hspace*{-20pt}\centering
		\includegraphics[width=1.3\textwidth]{emblem.png}
	\end{minipage}
	\begin{minipage}{0.82\textwidth}\small \textbf{
			\vspace*{-0.7ex}
			\hspace*{-10pt}\centerline{Министерство науки и высшего образования Российской Федерации}
			\vspace*{-0.7ex}
			\centerline{Федеральное государственное бюджетное образовательное учреждение }
			\vspace*{-0.7ex}
			\centerline{высшего образования}
			\vspace*{-0.7ex}
			\centerline{<<Московский государственный технический университет}
			\vspace*{-0.7ex}
			\centerline{имени Н.Э. Баумана}
			\vspace*{-0.7ex}
			\centerline{(национальный исследовательский университет)>>}
			\vspace*{-0.7ex}
			\centerline{(МГТУ им. Н.Э. Баумана)}}
	\end{minipage}

	\vspace{-2pt}
	\hspace{-34.5pt}\rule{\textwidth}{2.5pt}

	\vspace*{-20.3pt}
	\hspace{-34.5pt}\rule{\textwidth}{0.4pt}

	\vspace{0.5ex}
	\noindent \small ФАКУЛЬТЕТ\hspace{80pt} <<Информатика и системы управления>>

	\vspace*{-16pt}
	\hspace{35pt}\rule{0.855\textwidth}{0.4pt}

	\vspace{0.5ex}
	\noindent \small КАФЕДРА\hspace{50pt} <<Теоретическая информатика и компьютерные технологии>>

	\vspace*{-16pt}
	\hspace{25pt}\rule{0.875\textwidth}{0.4pt}


	\vspace{3em}

	\begin{center}
		\Large \bf{РАСЧЕТНО-ПОЯСНИТЕЛЬНАЯ ЗАПИСКА\\\textbf{\textit{К КУРСОВОЙ РАБОТЕ\\НА ТЕМУ:}} \\}
	\end{center}

	\vspace*{-6ex}
	\begin{center}
		\Large{\textit{\textbf{<<Сравнение методов коммуникации }}}

		\vspace*{-3ex}
		\rule{0.9\textwidth}{1.2pt}

		\vspace*{-0.2ex}
		\Large{\textit{\textbf{между процессами в среде Linux>>}}}
		\vspace*{-3ex}
		\vspace*{-0.2ex}
		\rule{0.9\textwidth}{1.2pt}

		\vspace*{-0.2ex}
		\rule{0.9\textwidth}{1.2pt}

		\vspace*{-0.2ex}
		\rule{0.9\textwidth}{1.2pt}

		\vspace*{-0.2ex}
		\rule{0.9\textwidth}{1.2pt}
	\end{center}

	\vspace{\fill}


	\newlength{\ML}
	\settowidth{\ML}{«\underline{\hspace{0.7cm}}» \underline{\hspace{2cm}}}

	\noindent Студент \underline{\text{ИУ9-51Б}} \hfill \underline{ \hspace{4cm}}\quad
	\underline{\parbox{4cm}{\centering Старовойтов А.И.}}

	\vspace{-1.8ex}
	\noindent\hspace{9ex}\scriptsize{(Группа)}\normalsize\hspace{170pt}\hspace{2ex}\scriptsize{(Подпись, дата)}\normalsize\hspace{30pt}\hspace{6ex}\scriptsize{(И.О. Фамилия)}\normalsize

	\bigskip

	\noindent Руководитель  \hfill \underline{\hspace{4cm}}\quad
	\underline{\parbox{4cm}{\centering Цалкович П.А.}}

	\vspace{-1.6ex}
	\noindent\hspace{13.5ex}\normalsize\hspace{170pt}\hspace{2ex}\scriptsize{(Подпись, дата)}\normalsize\hspace{30pt}\hspace{6ex}\scriptsize{(И.О. Фамилия)}\normalsize

	\bigskip

	\noindent Консультант\hfill \underline{\hspace{4cm}}\quad
	\underline{\hspace{4cm}}

	\vspace{-2ex}
	\noindent\hspace{13.5ex}\normalsize\hspace{170pt}\hspace{2ex}\scriptsize{(Подпись, дата)}\normalsize\hspace{30pt}\hspace{6ex}\scriptsize{(И.О. Фамилия)}\normalsize
	\vfill

	%\vspace{\fill}



	\begin{center}
		\textsl{2024 г.}
	\end{center}
\end{titlepage}

%\renewcommand{\ttdefault}{pcr}

\setlength{\tabcolsep}{3pt}
\newpage
\setcounter{page}{2}
%----------------------------------------------------------------------------
%                  ОТСЮДА --- СОБСТВЕННО ТЕКСТ
%----------------------------------------------------------------------------

\newpage
\renewcommand\contentsname{\hfill{\normalfont{СОДЕРЖАНИЕ}}\hfill}  %Оглавление
\tableofcontents
\newpage
\anonsection{ВВЕДЕНИЕ}  %Введение

Linux занимает доминирующее положение среди операционных систем в сегменте
вычислений на серверах\cite{OSMarketShare}, а также пользуется популярностью в
качестве десктопной системы у разработчиков\cite{DevelopersOS}. С учетом этого,
а также развития модульных, мультипроцессных программных систем, перед
разработчиками регулярно встает вопрос эффективной реализации обмена данными
между процессами в среде операционной системы Linux.

При проектировании системы использующей межпроцессное взаимодействие, должны
учитываться многие факторы, в том числе накладные расходы при применении таких
методов. В этом контексте актуальными являются бенчмарки, наглядно показывающие
скорость работы существующих интерфейсов обмена данными между процессами.

Целью данной работы является проведение сравнительного анализа интерфейса и
производительности существующих средств межпроцессного взаимодействия.

\section{Обзор предметной области}

Межпроцессное взаимодействие --- это обмен данными между потоками разных
процессов операционной системы, реалзизованный с помощью механизмов
предоставляемых операционной системой.

Механизмы межпроцессной коммуникации в Linux разделяют на три большие категории
по функциональному предназначению:\cite{kerrisk2010linux}

\begin{itemize}
  \item Коммуникационные \refImage{fig:communication_ipc_taxonomy}: фокусируются на
        обмене данными между процессами;
  \item Синхронизационные: для синхронизации действий между различными
        процессами;
  \item Сигналы: могут использоваться как для обмена данными, так и для
        синхронизации.
\end{itemize}

Как представлено на рисунке~\ref{fig:communication_ipc_taxonomy}, зачастую
несколько механизмов предоставляют схожий функционал. Это обусловлено рядом
причин, в их числе портирование интерфейсов между вариантами UNIX-подобных
систем и разработка новых интерфейсов для избавления от недостатков старых.

Но в некоторых случаях, механизмы предоставляющие схожий функционал, в
реальности имеют сильно различающиеся возможности. Например, каналы, в отличие
от FIFO, могут использоваться для коммуникации только между процессами, которые
имеют общего предка. А потоковые сокеты единственные в этой группе могут
использоваться для взаимодействия процессов на разных машинах по сети.

\begin{figure}[H]
  \centering
  \begin{minipage}[t]{\textwidth}
    \centering
    \begin{tikzpicture}[grow'=right,
      every node/.style={shape=rectangle,draw,align=center}]
      \node {Коммуника-\\ционные\\механизмы}
      [level distance=40mm, sibling distance=93mm, edge from parent fork right]
        child {
          node {Передача\\данных}
          [level distance=45mm, sibling distance=30mm, edge from parent fork right]
            child {
              node {Поток байтов}
              [shape=rounded rectangle, level distance=40mm, sibling distance=10mm, edge from parent fork right]
                child {
                  node [shape=rounded rectangle] {Каналы}
                }
                child {
                  node [shape=rounded rectangle] {FIFO}
                }
                child {
                  node [shape=rounded rectangle, yshift=-3mm] {Потоковые\\сокеты}
                }
            }
            child {
              node [shape=rounded rectangle] {Псевдо-\\терминал}
            }
            child {
              node {Сообщения}
              [level distance=40mm, sibling distance=25mm, edge from parent fork right]
                child {
                  node [shape=rounded rectangle] {Очереди\\сообщений\\System V}
                }
                child {
                  node [shape=rounded rectangle] {Очереди\\сообщений\\POSIX}
                }
                child {
                  % TODO: поменять название
                  node [shape=rounded rectangle] {datagram\\сокеты}
                }
            }
        }
        child {
          node {Разделяемая\\память} [level distance=45mm, sibling distance=25mm, edge from parent fork right]
          child {
            node [shape=rounded rectangle, xshift=15mm] {Разделяемая память System V}
          }
          child {
            node [shape=rounded rectangle, xshift=13mm] {Разделяемая память POSIX}
          }
          child {
            node {Отображаемая\\память}
            [level distance=45mm, sibling distance=25mm, edge from parent fork right]
              child {
                node [shape=rounded rectangle] {Анонимное\\отображение}
              }
              child {
                node [shape=rounded rectangle] {Отображение\\файла}
              }
          }
        };
    \end{tikzpicture}
  \end{minipage}
  \caption{Классификация коммуникационных механизмов межпроцессного взаимодействия в Linux.}
  \label{fig:communication_ipc_taxonomy}
\end{figure}

\subsection{Коммуникационные механизмы}

Коммуникационные механизмы являются основным способом передачи данных между
процессами в среде Linux. Их разнообразие представлено на
рисунке~\ref{fig:communication_ipc_taxonomy}.

Данная работа фокусируется именно на коммуникационных механизмах, т.к. именно
они представляют наибольший интерес в плане измерения накладных расходов при их
использовании.

Коммуникационные методы взаимодействия разделяют на две
категории:\cite{kerrisk2010linux}

\begin{itemize}
  \item \emph{Механизмы передачи данных}: их ключевое отличие заключается в
        операциях чтения и записи. Чтобы произвести обмен данными, требуется два
        системных вызова. Один процесс должен передать данные в буфер в ядре
        операционной системы, а другой затем считать их оттуда
        \refImage{fig:pipe_rw};
  \item \emph{Разделяемая память}: делает доступными другому процессу данные,
        которые были записаны в разделенный между этими процессами регион
        памяти. Это позволяет обмениваться данными без накладных расходов на
        системные вызовы, но требует дополнительной синхронизации между
        процессами записывающими и считывающими данные.
\end{itemize}

\subsubsection{Механизмы передачи данных}

Механизмы передачи данных, в свою очередь, разделяют на три подкатегории:

\begin{itemize}
  \item \emph{Поток байтов}: операции чтения и записи могут оперировать
        различным количеством байтов, данные записанные отдельными операциями не
        разделяются между собой;
  \item \emph{Сообщения}: отделенные друг от друга сообщения, операция чтения
        может считать сообщение только целиком, операция записи может записать
        сообщение тоже только целиком;
  \item \emph{Псевдотерминал}: используются для программ, ориентированных на
        использование в терминале, например: ssh, эмулятор терминала.
\end{itemize}

При этом, выделяют два основных отличия механизмов передачи данных от
разделяемой памяти:

\begin{itemize}
  \item данные нельзя прочитать 2 раза, т.е. после чтения данные удаляются из
        буфера операционной системы;
  \item синхронизация между процессами считывающими и записывающими данные
        производится автоматически.
\end{itemize}

\subsubsection{Разделяемая память}

Разделяемую память можно описать как более низкоуровневый инструмент
межпроцессной коммуникации, чем передача данных. Преимущество низких накладных
расходов при использовании этого инструмента сглаживается необходимостью
дополнительной ручной синхронизации, что помимо прочего, усложняет
пользовательский код, но делает его более гибким.

\subsubsection{Параметры для сравнения механизмов межпроцессной коммуникации}

При выборе механизма межпроцессной коммуникации, разработчики опираются на ряд
факторов, таких как:

\begin{itemize}
  \item \emph{Интерфейс взаимодействия}: значительная часть механизмов обмена
        данными между процессами в Linux используют традиционные для
        POSIX-систем файловые дескрипторы, что делает интерфейс для работы с
        ними единообразным и взаимозаменяемым;
  \item \emph{Функциональность}: поток байт или сообщения фиксированной длины,
        поддержка мультиплексирования ввода-вывода, поддержка уведомлений с
        помощью сигналов, поддержка нескольких слушателей и т.п.
  \item Возможность передачи данных по сети;
  \item Совместимость с другими реализациями UNIX;
  \item Управление доступом;
  \item Персистентность;
  \item Производительность.

\end{itemize}

\subsection{Каналы}

В UNIX-подобных операционных системах каналы являются первым реализованным
методом межпроцессного взаимодействия. Впервые они появлись в 1970-х
годах.\cite{kerrisk2010linux} Каналы естественным образом реализуют философию
UNIX, позволяя использовать вывод одной команды, как ввод для другой команды,
что используется командной оболочкой для составления конвееров обработки данных.

Интерфейс канала предполагает передачу данных в одну сторону. Для
двунаправленной передачи данных создают два канала.

Категория ``Поток байтов'', к которой относят каналы, означает, что данные не
отделяются друг от друга при нескольких вызовах операции записи. Операцией
чтения можно считывать из канала любое число байт, но только в той же
последовательности, что они были записаны.

Разделенные друг от друга сообщения можно передавать через канал, но это требует
дополнительной поддержки на стороне кода приложения. Например, можно отправлять
и считывать сообщения фиксированной длины, либо перед телом сообщения отправлять
его размер.

В ядре Linux существует константа \verb|PIPE_BUF|. Операции записи, которые
передают менее \verb|PIPE_BUF| байт, всегда происходят атомарно. Кроме того,
операции записи не блокируются, если в буфере операционной системы достаточно
места для сохранения данных, которые еще не считаны читающим процессом. Начиная
с версии 2.6.11, размер буфера установлен в 65'536 байт, и может быть расширен
до 1'048'576 байт и более, в зависимости от настроек системы.

Каналы создаются с помощью системного вызова \verb|pipe()|, который создает два
открытых файловых дескриптора для чтения и записи соответственно. Операция
чтения производится с помощью \verb|read()|, а записи с помощью \verb|write|.
Это стандартный способ работы с файловыми дескрипторами в Linux.

Чтобы коммуницировать между процессами с помощью канала, необходимо создать
канал с помощью вызова \verb|pipe()|, а затем создать дочерний процесс с помощью
\verb|fork()|. При этом дочерний процесс наследует открытые файловые дескрипторы
родительского процесса. Затем, дочернему и родительскому процессу нужно закрыть
файловые дескрипторы соответствующие противоположным концам канала, т.е.
читающему процессу нужно закрыть дескриптор для записи, а записывающему заркыть
дескриптор для чтения. Это позволит читающему процессу обнаружить, когда
записывающий процесс закончит работу с каналом и закроет свой десприптор для
записи. После этих манипуляций, процессы могут использовать операции чтения и
записи на соответствующих файловых десприпторах для коммуникации через канал.

\subsection{FIFO}

FIFO реализует тот же концепт (``Поток байт''), что и каналы, но допускает
взаимодействие несвязанных через \verb|fork()| процессов. Это достигается тем,
что FIFO имеет название в файловой системе и имеет интерфейс обычного файла.
Таким образом, FIFO иногда называют именованным каналом.

FIFO создается с помощью вызова \verb|mkfifo()|, в который передается путь в
файловой системе. Этот вызов создает файл FIFO по переданному пути, после чего
процессы могут открыть файловые десприпторы для чтения или записи используя тот
же путь, применяя для этого системный вызов \verb|open()|.

Кроме создания, интерфейс для работы с именованными каналами не отличается от
обычных каналов, но предоставляет большую гибкость, т.к. процессы не должны быть
связанными.

\section{Разработка приложения}

Приложение будет являться клиентской библиотекой, предостовляющей возможности
отрисовки графов с помощью нескольких силовых алгоритмов, рассмотренных выше.
Кроме функционала для отрисовки, необходимо разработать интерфейсы и модели для
взаимодействия с модулем для работы с API социальных сетей. Таким образом,
станет возможно отделить логику визуализации от логики получения и
преобразования данных пользователей и их связей.

\subsection{Архитектура приложения}


\begin{enumerate}
	\item{Модуль сбора данных.
	      Этот модуль представляет собой первый этап обработки данных в библиотеке и отвечает за сбор информации из социальной сети. Данный модуль является заменяемым компонентом, что позволяет интегрировать уникальные методы сбора данных для различных социальных сетей, сохраняя при этом совместимость с остальными модулями. Задачи модуля включают:
	      \begin{itemize}
		      \item{Получение данных: модуль осуществляет запросы к API социальной сети для получения информации о пользователях, связях между ними и других существенных данных;}
		      \item{Преобразование данных: полученная информация преобразуется в формат, легко обрабатываемый другими частями приложения. Это может включать в себя фильтрацию ненужной информации            и преобразование их в структуры, более удобные для последующей обработки.}
	      \end{itemize}
	      }
	\item{Модуль обработки данных.
	      Этот модуль выполняет обработку данных, полученных от модуля сбора, и преобразует их в графовую модель. Задачи этого модуля включают:

	      \begin{itemize}
		      \item Формирование графовой структуры: модуль создает структуры данных, представляющие собой граф с определенными координатами вершин. Эти данные готовы к передаче модулю визуализации;
		      \item Применение силовых алгоритмов: К модулю поступают данные о вершинах графа, и на них применяются силовые алгоритмы.
	      \end{itemize}


	      }
	\item{Модуль визуализации занимается отображением графа, обработанного силовым алгоритмом. Его задачи включают:

	      \begin{itemize}
		      \item Отображение вершин и рёбер: Модуль визуализации отображает граф, используя координаты вершин и информацию о связях между ними. Это включает в себя определение визуального представления вершин и рёбер;
		      \item Интерактивность: Предоставление пользователю возможности взаимодействия с визуализацией. Это может включать в себя приближение/удаление, перемещение вершин, отображение дополнительной информации при наведении и т. д.
	      \end{itemize}
	      }

\end{enumerate}

Эти три модуля в совокупности обеспечивают работу от сбора данных из социальной сети до визуализации обработанного графа с применением силовых алгоритмов.

Схема работы приложения представлена на рисунке \ref{fig:app_scheme}


\begin{figure}[H]
	\centering
	\begin{minipage}[t]{.9\textwidth}
		\centering
		\includegraphics[width=.9\textwidth]{./imgs/app_scheme.png}
	\end{minipage}
	\caption{Схема взаимодействия модулей приложения.}
	\label{fig:app_scheme}
\end{figure}

\subsection{Выбор инструментов разработки}

Для разработки библиотеки был выбран язык TypeScript \cite{TS} из-за следующих его особенностей:

\begin{enumerate}
	\item Статическая Типизация. TypeScript предлагает статическую типизацию, что позволяет выявлять и предотвращать множество ошибок на этапе разработки, что в свою очередь способствует более безопасной разработке.
	\item Расширенная Поддержка ООП. TypeScript, как расширение JavaScript, обладает мощными возможностями объектно-ориентированного программирования. Это особенно важно при создании таких структур данных как графы. Классы и интерфейсы TypeScript облегчают организацию кода и создание модульных и поддерживаемых компонентов.
	\item Экосистема JavaScript. TypeScript является надмножеством JavaScript, что означает полную совместимость с существующей JavaScript-экосистемой. Это позволяет использовать существующие библиотеки и инструменты в разработке библиотеки, обеспечивая гибкость и расширяемость проекта.
	\item Поддержка Современных Стандартов. TypeScript активно поддерживается и обновляется, включая поддержку последних стандартов ECMAScript. Это важно для использования новых возможностей языка.
	\item Инструменты Для Рефакторинга и Анализа Кода. TypeScript обеспечивает богатый набор инструментов для рефакторинга кода и анализа его качества. Это упрощает процессы поддержки и развития библиотеки.
\end{enumerate}


После выборая ЯП, следующим шагом является выбор подходящей библиотеки визуализации. Библиотеки для визуализации предоставляют ряд инструментов, необходимых для эффективного представления и взаимодействия с данными графов. Среди всех продуктов больше всего выделяются: D3.js, Cytoscape.js, Vis.js. Среди всех перечисленных вариантов для данного проекта лучше всего подходит Cytoscape.js \cite{Cytoscapejs}, т.к.
специализируется именно на визуализации графов, что делает её более оптимизированной для конкретно этого вида задач. Кроме того, важны следующие факторы:

\begin{enumerate}
	\item Cytoscape.js предоставляет обширный функционал для работы с графами, включая поддержку различных макетов, стилей, анимаций и обработки событий. Это обеспечивает гибкость в адаптации библиотеки под уникальные требования проекта;

	\item Библиотека Cytoscape.js оптимизирована для эффективной отрисовки крупных графов, что соответствует требованиям проекта по визуализации сложных структур данных. Оптимизированные алгоритмы позволяют обеспечить высокую производительность даже при работе с большим количеством элементов;


	\item Cytoscape.js находится под активной поддержкой сообщества разработчиков. Регулярные обновления и внимание к запросам сообщества гарантируют наличие актуальных версий библиотеки и устранение возможных проблем.

\end{enumerate}

\subsection{Разработка моделей приложения}


Для эффективной работы с данными социальных сетей в приложении необходимо разработать две ключевые модели --- модель пользователя и модель графа. Эти модели будут играть центральную роль в преобразовании и структурировании данных для последующей визуализации и взаимодействия.

Модель пользователя представляет абстракцию данных о каждом участнике социальной сети. Сервер передает данные в формате JSON \cite{JSON}, который необходимо преобразовать в объекты языка программирования для дальнейшего удобства взаимодействия.

Поля модели поользователя:

\begin{itemize}
	\item Идентификатор: уникальный идентификатор пользователя в социальной сети;
	\item Имя и Фамилия: информация о имени и фамилии пользователя;
	\item Фотография: ссылка на фотографию пользователя.
\end{itemize}


Модель пользователя служит для:

\begin{itemize}
	\item Преобразования данных из формата JSON в объекты языка программирования;
	\item Хранения основных свойств пользователя для дальнейшего использования в приложении.
\end{itemize}

Модель графа необходима для структурирования информации о связях между пользователями и предоставления основы для визуализации графа социальной сети. Модель графа включает в себя модель вершины, которая является расширением модели пользователя.


Объекты модели графа:

\begin{itemize}
	\item Модель вершины является расширенной моделью пользователя, содержащей дополнительную информацию о расположении пользователя в графе;
	\item Список инцидентности: Информация о взаимосвязях между пользователями.
\end{itemize}

Модель графа выполняет следующие функции:

\begin{itemize}
	\item Организация и хранение связей между пользователями;
	\item Возможность определения расположения каждого пользователя в графе;
	\item Структурирование данных для визуализации и взаимодействия с графом социальной сети.
\end{itemize}

Разработанные модели обеспечивают необходимый фундамент для эффективной работы с данными социальных сетей. Модель пользователя обеспечивает удобное представление информации о каждом участнике сети, в то время как модель графа структурирует и предоставляет основу для визуализации связей между пользователями.

\subsection{Разработка силовых алгоритмов}

Для использования в разрабатываемой библиотеке были выбраны алгоритмы Фрюхтермана-Рейнгольда и Камады-Кавай. Алгоритма Идеса, хотя и эффективен для небольших графов, но уступает выбранным алгоритмам в практической применимости. Алгоритм Идеса имеет скорее теоретическую ценность, так как он является одним из первых силовых алгоритмов, на основе которого развиваются более совершенные методы в данной области.

Указанные алгоритмы оперируют двумерными векторами, поэтому в первую очередь небходимо разработать класс двумерного вектора со следующими операциями:

\begin{itemize}
	\item сложение
	\item вычитание
	\item умножение на скаляр
	\item получение евклидовой нормы
	\item нормализация
\end{itemize}




\section{Реализация приложения}

Для выполнения поставленной задачи необходимо реализовать три
модуля приложения отвечающих за сбор, обработку и пребразование данных cоциальной сети.

Разработка проекта осуществлялась в редакторе кода Neovim \cite{NVIM}. Данный редактор может быть использван в качестве IDE благодаря обширной библиотеке расширений.
Расширение <<typescript-language-server>> \cite{TSLSP} предоставляет умное автодополнение, инструменты для рефакторинга и анализа кода --- возможности, необходимые для быстрой и
эффективной разработки на typescript.

Для сборки проекта используется  инструмент сборки для веб-приложений под названием Parcel \cite{PARCEL}.
Он предоставляет простой и быстрый способ управления зависимостями и создания оптимизированных пакетов кода для развертывания веб-приложений.

\subsection{Особенности реализации модуля сбора данных}

Для сбора данных была выбрана социальная сеть <<Вконтакте>>. Данная соцсеть предоставляет публичный API \cite{VKAPI}, с помощью которого можно получить данные о пользователях и их связях. Для обращения к API был реализован отдельный модуль, содержащий следующие компоненты:

\begin{itemize}
	\item Класс VkAPI, предоставляющий методы для получения всех друзей определенного пользователя, и метод для получение связей внутри некоторой группы пользователей(реализация класса представлен на листинге \ref{lst:vkapi});
	\item Тип данных User, небходимый для преобразование ответа из формата JSON в объект языка(реализация представлена на листинге \ref{lst:user_model});
	\item Набор типов представляющих собой структуры ответов на запросы к API(реализация представлена на листинге \ref{lst:responses})
	      \begin{enumerate}
		      \item Тип ErrorResponse --- содержит сообщение об ошибке в случае запроса, который завершился неуспешно.
		      \item Тип GetFriendsResponse --- представляет собой успешный ответ на запрос всех друзей пользователя. Содержит поле, указывающие на количество друзей, и массив со структурами User, описывающий всех друзей запрашиваемого пользователя.
		      \item Тип GetMutualResponse --- представляет собой успешный ответ на запрос о связях внутри группы. Данный тип является массивом объектов, которые описывают какие связи имеет каждый пользователь группы.
	      \end{enumerate}

\end{itemize}


\subsection{Особенности реализации алгоритма Фрюхтермана-Рейнгольда}

Для реализации функциональность алгоритма Фрюхтермана-Рейнгольдаа был написан отдельный модуль, который состоит из двух локальных и одной экспортируемой функий
, каждая из которых выполняет свою задачу в рамках алгоритма.

\begin{itemize}
	\item Функция подсчета отталкивающей силы (представлена на листинге \ref{lst:rep_force});
	\item Функция подсчета притягивающей силы (представлена на листинге \ref{lst:attr_force});
	\item Функция реализующая главный алгоритм (представлена на листинге \ref{lst:fr_alg}). Она управляет взаимодействием отталкивающих и притягивающих сил, а также обновляет координаты узлов для достижения оптимального распределения;

\end{itemize}


Данный модуль предоставляет высокоуровневый интерфейс, с помощью которого можно преобразовать граф согласно алгоритму Фрюхтермана-Рейнгольда. Локальные функции (подсчет отталкивающей и притягивающей сил) не видны из других модулей, что обеспечивает инкапсуляцию и изоляцию внутренних деталей алгоритма.



\subsection{Особенности реализации алгоритма Камады-Кавай}

Посколько алгоритм Камады-Кавай является более сложным в реализации, для него был написан отдельный класс, предоставляющий методы для преобразования графа. Для реализации данного алгоритма были реализованы следующие функции и методы:

\begin{itemize}
	\item Алгоритм Флойда-Уоршела для поиска кратчайших путей в графе (представлен на листинге \ref{lst:fw_alg});
	\item Метод поиска значения энергии вершины (представлен на листинге \ref{lst:vertexen});
	\item Метод поиска вершины с максимальным значением энергии (представлен на листинге \ref{lst:maxen});
	\item Метод вычисления нового положения вершины (представлен на листинге \ref{lst:nextpos});
	\item Главный метод, реализующий основной алгоритм (представлен на листинге \ref{lst:kk_alg});
\end{itemize}

Для применения к графу алгоритма Камады-Кавай небходимо создать объект KamadaKawai, передав в конструктор граф, к которому должен применяться алгоритм. Далее,
ипользуя вызвов метода run, можно получить граф с оптимальной компановкой.



\subsection{Описание работы приложения}

Для работы с модулем сбора данных социальной сети <<Вконтакте>> небходимо получить специальный ключ, который будет отправляться с каждый обращением к API. Если данный ключ получен, то работа приложения строится следущим образом:

\begin{enumerate}
	\item Небходимо создать объкт VkAPI, передав в конструктор полученный ранее ключ;
	\item Для рассмотрения связей внутри группы друзей некоторого пользователя небходим получить его уникальный цифрофой идентификатор.
	\item Поскольку данное приложение полностью выполняется в браузере, то при запросе к API социальной сети <<Вконтакте>> возникают проблемы связанные с CORS \cite{CORS}. Для решения данных проблем небходимо воспользоваться протоколом JSONP \cite{JSONP}, согласно котору для выполнения запроса нужно создать элемент script с свойством src равным тому адресу, к которому небходимо совершить запрос. Также данный адрес должен содержать название функции обратного вызова, которая будет вызвана в момент возвращения ответа. Каждый из методов для запроса данных объекта VkAPI содержит параметр, который является функцией, которая должна будет вызваться при возвращение ответа.
	\item После получения всех данных небходимо сконструировать объект графа, передав в конструктор массив пользователей графа и матрицу смежности.
	\item К полученному графу небходимо применить один из доступных алгоритмов, и передать преобразованный граф в функцию draw(представлена на листинге \ref{lst:draw}), вместе с предварительно созданным объектом типа cytoscape.Core()
\end{enumerate}

Запуск приложения производиться согласно рисунку \ref{fig:run_app}. После успешной сборки в консоли появится адрес, а также откроется окно браузера с визуализацией графа. Примеры визуализации графа из 60 пользователей с помощью алгоритмов Фрюхтермана-Рейнгольда и Камады-Кавай представлены на рисунках \ref{fig:fr_result} и  \ref{fig:kk_result} соответственно.

\begin{figure}[H]
	\centering
	\begin{minipage}[t]{.8\textwidth}
		\centering
		\includegraphics[width=.8\textwidth]{./imgs/run_app.png}
	\end{minipage}
	\caption{Пример запуска приложения.}
	\label{fig:run_app}
\end{figure}


\begin{figure}[H]
	\centering
	\begin{minipage}[t]{.7\textwidth}
		\centering
		\includegraphics[width=.7\textwidth]{./imgs/fr.png}
	\end{minipage}
	\caption{Результат визуализации с помощь алгоритма Фрюхтермана-Рейнгольда.}
	\label{fig:fr_result}
\end{figure}


\begin{figure}[H]
	\centering
	\begin{minipage}[t]{.9\textwidth}
		\centering
		\includegraphics[width=.9\textwidth]{./imgs/kk.png}
	\end{minipage}
	\caption{Результат визуализации с помощью алгоритма Камады-Кавай.}
	\label{fig:kk_result}
\end{figure}

\section{Тестирование}

Для тестирования приложения проведем сравнения работы программы, с результатми полученными программой graphviz \cite{GV}. Сравнение будет производится на различных, с точки зрения структуры, графах.

\subsection{Тестирование на полном графе}

Результат работы программы на небольшом полном графе представлен на рисунке \ref{fig:full_graph_res}, пункты а и б. Результат визуализации с помощью graphviz представлен на том же рисунке, пункт в. Как можно увидеть, укладки графов идентичны, с точность до поворота.

\begin{figure}[H]
	\centering
	\begin{minipage}[t]{.55\textwidth}
		\centering
		\includegraphics[width=0.75\textwidth]{./imgs/fg_k6.png}
		\caption*{а) алгоритм Фрюхтермана-Рейнгольда.}
	\end{minipage}
	\noindent
	\begin{minipage}[t]{.45\linewidth}
		\centering
		\includegraphics[width=\textwidth]{./imgs/kk_k6.png}
		\caption*{б) алгоритм Камады-Кавай.}
	\end{minipage}
	\begin{minipage}[t]{.45\textwidth}
		\centering
		\includegraphics[width=\linewidth]{./imgs/k6_gv.png}
		\caption*{в) программа Graphviz.}
	\end{minipage}
	\caption{Результат работы программы на полном графе.}
	\label{fig:full_graph_res}
\end{figure}

\subsection{Тестирование на плотном графе}


Результат работы программы на небольшом плотном графе представлен на рисунке \ref{fig:small_dense_res}, пункты а и б. Результат визуализации с помощью graphviz представлен на том же рисунке, пункт в. Результаты визуализации практически идентичны.

\begin{figure}[H]
	\centering
	\begin{minipage}[t]{.35\textwidth}
		\centering
		\includegraphics[width=\linewidth]{./imgs/fg_small_dense.png}
		\caption*{а) алгоритм Фрюхтермана-Рейнгольда.}
	\end{minipage}
	\noindent
	\begin{minipage}[t]{.40\textwidth}
		\centering
		\includegraphics[width=\linewidth]{./imgs/kk_small_dense.png}
		\caption*{б) алгоритм Камады-Кавай.}
	\end{minipage}
	\begin{minipage}[t]{.55\textwidth}
		\centering
		\includegraphics[width=\linewidth]{./imgs/small_dense_gv.png}
		\caption*{в) программа Graphviz.}
	\end{minipage}
	\caption{Результат работы программы на небольшом плотном графе.}
	\label{fig:small_dense_res}
\end{figure}



\subsection{Тестирование на бинарном дереве}

Результат работы программы на графе, не содержащем циклов,  и  каждая вершина которого имеет не более трех смежных ребер, представлен на рисунке \ref{fig:bin_tree_res}, пункты а и б. Как можно увидеть, результататы полученные разработанной программой, и с помощью graphviz, практически совпадают.

\begin{figure}[H]
	\centering
	\begin{minipage}[t]{.43\textwidth}
		\centering
		\includegraphics[width=\linewidth]{./imgs/fr_btree.png}
		\caption*{а) алгоритм Фрюхтермана-Рейнгольда.}
	\end{minipage}
	\noindent
	\begin{minipage}[t]{.56\linewidth}
		\centering
		\includegraphics[width=0.75\textwidth]{./imgs/kk_btree.png}
		\caption*{б) алгоритм Камады-Кавай.}
	\end{minipage}
	\begin{minipage}[t]{.55\textwidth}
		\centering
		\includegraphics[width=\linewidth]{./imgs/bin_tree_gv.png}
		\caption*{в) программа Graphviz.}
	\end{minipage}
	\caption{Результат работы программы на бинарном дереве.}
	\label{fig:bin_tree_res}
\end{figure}

\subsection{Тестирование на k-арном дереве}

Результат работы программы на графе, являющимся k-арным деревом при k = 4, представлен на рисунке \ref{fig:quad_tree_res}, пункты а и б. Как можно увидеть, результататы полученные разработанной программой, и с помощью graphviz, практически совпадают.

\begin{figure}[H]
	\centering
	\begin{minipage}[t]{.48\textwidth}
		\centering
		\includegraphics[width=\linewidth]{./imgs/fr_quad_tree.png}
		\caption*{а) алгоритм Фрюхтермана-Рейнгольда.}
	\end{minipage}
	\noindent
	\begin{minipage}[t]{.50\textwidth}
		\centering
		\includegraphics[width=\linewidth]{./imgs/kk_quad_tree.png}
		\caption*{б) алгоритм Камады-Кавай.}
	\end{minipage}
	\begin{minipage}[t]{.55\textwidth}
		\centering
		\includegraphics[width=\linewidth]{./imgs/quad_tree_gv.png}
		\caption*{в) программа Graphviz.}
	\end{minipage}
	\caption{Результат работы программы на k-арном дереве, при k=4.}
	\label{fig:quad_tree_res}
\end{figure}


\subsection{Результаты тестирования}

В результате тестирования программы на различных структурах графов и сравнения её результатов с визуализацией, полученной с использованием Graphviz, можно утверждать о правильности работы программы. Программа продемонстрировала согласованность и точность в создании расположений для разнообразных типов графов, включая полные, плотные и графы с ограничением на количество смежных рёбер у вершин. На каждом этапе тестирования результаты работы программы оказывались практически идентичными визуализации, полученной с помощью Graphviz.

Обнаруженные расхождения между результатами программы и Graphviz были минимальными и, в большинстве случаев, могут быть объяснены допустимыми вариациями в укладке графов.

Таким образом, результаты тестирования позволяют сделать вывод о правильности работы разработанной программы и её способности генерировать расположения графов, имеющих различные характеристики.


\newpage
\anonsection{ЗАКЛЮЧЕНИЕ}  %Заключение


В ходе выполнения курсовой работы был проведен анализ способов визуализации связей пользователей в социальных сетях. Рассмотрены различные алгоритмы для визуализации графовых данных с учетом особенностей структуры социальных связей.

Основной целью данной курсовой работы была реализация программы для визуализации связей пользователей в социальной сети. Результаты выполнения задачи демонстрируют удовлетворительное качество работы программы, что подтверждается сравнением с результатами визуализации, полученными с использованием Graphviz. Разработанное программное решение представляет собой инструмент для анализа и изучения структуры социальных связей в различных сообществах.

В заключение, несмотря на достигнутые положительные результаты, существует потенциал для дальнейшего улучшения программы. Это включает в себя оптимизацию механизмов обработки и анализа данных, адаптацию программы для различных типов социальных сетей, а также повышение удобства использования в различных сценариях исследования социальных взаимосвязей.

\newpage
\renewcommand\refname{СПИСОК ИСПОЛЬЗОВАННЫХ ИСТОЧНИКОВ}
% Список литературы
\clearpage
\phantomsection
\addcontentsline{toc}{section}{\protect\numberline{}\refname}
%\bibliographystyle{ugost2008s}  %utf8gost71u.bst} %utf8gost705u} %gost2008s}
{\catcode`"\active\def"{\relax}
%\bibliography{biblio} %здесь ничего не меняем, кроме, возможно, имени bib-файла
\printbibliography
}
\newpage
\settocdepth{section}
\anonsection{ПРИЛОЖЕНИЕ А}
\vspace{-30pt}



%
% \begin{listing}[H]
%
%     \caption{Модель двумерного вектора}
%     \label{lst:vector_model}
%     \inputminted[breaklines, frame=single,fontsize = \footnotesize, linenos, xleftmargin = 1.5em]{typescript}{./listings/vector.ts}
% \end{listing}
%
% \begin{listing}[H]
%     \caption{Операции над векторами}
%     \label{lst:vector_ops}
%     \inputminted[breaklines, frame=single,fontsize = \footnotesize, linenos, xleftmargin = 1.5em]{typescript}{./listings/vector_ops.ts}
% \end{listing}

\begin{listing}[H]
	\caption{Модель пользователя}
	\label{lst:user_model}
	\inputminted[style=bw, frame=single,fontsize = \footnotesize, linenos=false, xleftmargin = 1.5em]{typescript}{./listings/user.ts}
\end{listing}

% \begin{listing}
% \caption{Модель вершины}
% \label{lst:vertex_model}
%   \inputminted[breaklines, frame=single,fontsize = \footnotesize, linenos, xleftmargin = 1.5em]{typescript}{./listings/vertex.ts}
% \end{listing}
%
% \begin{listing}
% \caption{Модель графа}
% \label{lst:graph_model}
%   \inputminted[breaklines, frame=single,fontsize = \footnotesize, linenos, xleftmargin = 1.5em]{typescript}{./listings/graph.ts}
% \end{listing}
%

\begin{listing}[H]
	\caption{Структуры ответов}
	\label{lst:responses}
	\inputminted[style=bw, breaklines, frame=single,fontsize = \footnotesize, linenos=false,, xleftmargin = 1.5em]{typescript}{./listings/responses.ts}
\end{listing}

\begin{listing}[H]
	\caption{Реализация класса VkAPI}
	\label{lst:vkapi}
	\inputminted[style=bw, breaklines, frame=single,fontsize = \footnotesize, linenos=false,, xleftmargin = 1.5em]{typescript}{./listings/vkapi.ts}
\end{listing}


\begin{listing}[H]
	\caption{Реализация алгоритма Фрюхтермана-Рейнгольда}
	\label{lst:fr_alg}
	\inputminted[style=bw, breaklines, frame=single,fontsize = \footnotesize, linenos=false,, xleftmargin = 1.5em]{typescript}{./listings/fr.ts}
\end{listing}

\begin{listing}[H]
	\caption{Реализация функции нахождения отталкивающей силы}
	\label{lst:rep_force}
	\inputminted[style=bw, breaklines, frame=single,fontsize = \footnotesize, linenos=false,, xleftmargin = 1.5em]{typescript}{./listings/rep.ts}
\end{listing}

\begin{listing}[H]
	\caption{Реализация функции нахождения притягивающей силы}
	\label{lst:attr_force}
	\inputminted[style=bw, breaklines, frame=single,fontsize = \footnotesize, linenos=false,, xleftmargin = 1.5em]{typescript}{./listings/rep.ts}
\end{listing}


\begin{listing}[H]
	\caption{Реализация алгоритма Флойда-Уоршела}
	\label{lst:fw_alg}
	\inputminted[style=bw, breaklines, frame=single,fontsize = \footnotesize, linenos=false,, xleftmargin = 1.5em]{typescript}{./listings/fw.ts}
\end{listing}


\begin{listing}[H]
	\caption{Метод поиска значения энергии вершины}
	\label{lst:vertexen}
	\inputminted[style=bw, breaklines, frame=single,fontsize = \footnotesize, linenos=false,, xleftmargin = 1.5em]{typescript}{./listings/vertexen.ts}
\end{listing}

\begin{listing}[H]
	\caption{Метод поиска вершины с максимальным значением энергии}
	\label{lst:maxen}
	\inputminted[style=bw, breaklines, frame=single,fontsize = \footnotesize, linenos=false,, xleftmargin = 1.5em]{typescript}{./listings/maxen.ts}
\end{listing}

\begin{listing}[H]
	\caption{Метод поиска нового положения вершины}
	\label{lst:nextpos}
	\inputminted[style=bw, breaklines, frame=single,fontsize = \footnotesize, linenos=false,, xleftmargin = 1.5em]{typescript}{./listings/nextpos.ts}
\end{listing}

\begin{listing}[H]
	\caption{Реализация алгоритма Камады-Кавай}
	\label{lst:kk_alg}
	\inputminted[style=bw, breaklines, frame=single,fontsize = \footnotesize, linenos=false, xleftmargin = 1.5em]{typescript}{./listings/kk.ts}
\end{listing}

\begin{listing}[H]
	\caption{Реализация функции draw}
	\label{lst:draw}
	\inputminted[style=bw, breaklines, frame=single,fontsize = \footnotesize, linenos=false, xleftmargin = 1.5em]{typescript}{./listings/draw.ts}
\end{listing}



\end{document}
